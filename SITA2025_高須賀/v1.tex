% 第42回情報理論とその応用シンポジウム 予稿集 原稿様式
% e-pTeX, Version 3.14159265-p3.7.1-161114-2.6 (utf8.euc) (TeX Live 2017/Debian) (preloaded format=platex)
% 本文: 日本語
\documentclass{jarticle}
\usepackage{sita2021}
\usepackage{amsmath,amssymb,amsthm}
\usepackage{mathtools}
\usepackage[dvipdfmx]{graphicx}
\usepackage{bm}
\usepackage{bbm}
\usepackage{multicol}
\usepackage{multirow}
\usepackage{lscape}
\usepackage{comment}
\usepackage{longtable}
\usepackage{url}
\usepackage[dvipdfmx]{color}
\usepackage{float}
%%%%% Theorem environment 定理環境 %%%%%%%%%%%%%%%%%%%%%%%%%%%%%%%%%%%%%%%%%%%%
\theoremstyle{definition}
%\theoremsymbol{\ensuremath{\Box}}
\newtheorem{theorem}{定理}
\newtheorem{prop}{命題}
\newtheorem{lemma}{補題}
\newtheorem{cor}{系}
\newtheorem{example}{例}
\newtheorem{definition}{定義}
\newtheorem{rem}{注意}
\newtheorem{guide}{参考}
\newtheorem{assumption}{仮定}
\renewcommand\proofname{\bf 証明}
\newcommand{\halfeq}[2][0.90\linewidth]{%
  \begin{equation}
    \makebox[\linewidth][l]{\resizebox{#1}{!}{\ensuremath{#2}}}%
  \end{equation}
}
\renewcommand{\baselinestretch}{.9}

\title{
  %和文の論文題目
  構造類似度指標の統合によるSOM気圧配置分類法の一考察\\
  %英文の論文題目
  Synoptic Pattern Classification via Integrated Structural Similarity Metrics in SOM
}
\author{
  %和文の第一著者名
  高須賀匠
  \thanks{ %和文の所属と住所
    〒940-2188 新潟県長岡市上富岡町1603-1 長岡技術科学大学,
    %英文の所属と住所
    Nagaoka University of Technology, 1603-1 Kamitomioka-machi, Nagaoka, Niigata 940-2188, Japan. E-mail: s233319@stn.nagaokaut.ac.jp.
  }\\
  %英文の第一著者名
  Takumi Takasuka
  \and
  %第二著者名(和文)
  高野雄紀
  \thanks{
    %和文の所属と住所
    〒305-0052 茨城県つくば市長峰1-1 気象研究所,
    %英文の所属と住所
    Meteorological Research Institute, 1-1 Nagamine, Tsukuba, Ibaraki 305-0052, Japan.
  }\\
  %第二著者名(英文)
  Yuki H. Takano
  \and
  渡邊正太郎
  \thanks {
    %和文の所属と住所
    〒107-0052 東京都港区赤坂5-4-7 The HEXAGON 5F 株式会社ウェザーマップ,
    %英文の所属と住所
    Weather Map Co., Ltd., The HEXAGON 5F, 5-4-7 Akasaka, Minato-ku, Tokyo 107-0052, Japan.
  }\\
  Shotaro Watanabe
  \and
  雲居玄道
  \footnotemark[1]\\
  Gendo Kumoi
}
\abstract{
This study compares distance metrics for synoptic pattern classification using Self-Organizing Maps (SOM). We implemented Batch-SOM with eight metrics: Euclidean (EUC), Structural Similarity Index (5x5 window, SSIM5), Teweles-Wobus score (S1), and their fusion (S1+SSIM5). Using sea level pressure data from Japan (1991–2000), we evaluated performance via Macro-averaged Recall and medoid representation quality. Results showed S1+SSIM5 had superior generalization. We also introduced medoid and true medoid representations to reduce blurring from centroid averaging, enhancing the interpretability of SOM output maps for operational use, bridging advances in structural similarity metrics.}
\keywords{
  Self-Organizing Map (SOM), synoptic pattern classification, structural similarity metrics
}
\begin{document}
\maketitle

\section{はじめに}
総観規模の気圧配置パターンを分類する技術は,気象予報や気候システムの理解,さらには防災・減災といった多様な分野で基盤となるものである.従来の専門家による主観的な分類は,その知見を直接反映できる利点があるものの,多大な労力や再現性,拡張性に課題があった.そのため,客観的かつ自動的な分類手法が広く研究されている.

教師あり学習の分野では,木村ら\cite{木村広希2009サポートベクターマシンを用いた気圧配置検出手法の提案}がサポートベクターマシン(SVM)を用いて「冬型」や「南高北低型」などを自動検出し,実用的な検索システムを構築した.しかし,このアプローチにはラベル付けのコストや,主観のばらつき(ノイズ)による学習データの品質の限界といった課題が指摘されている.

一方,教師なし学習では,自己組織化マップ(SOM)が総観パターンの持つ複雑な非線形構造を可視化・圧縮する有効な手段として用いられてきた\cite{philippopoulos2014performance,jiang2013classification}.国内の研究では,筆者ら\cite{takasuka2024}がBatch-SOMを用いて日本周辺の気圧配置をクラスタリングし,1kmメッシュの気象データと関連付ける研究成果を示している.

多くの従来研究ではユークリッド距離を基本としているが,勾配・形状・位置といった「構造」の評価には限界がある.これを改善するため,S-SOM\cite{doan2021s}では構造類似度指標(SSIM)をBMU探索に導入し,トポロジーの保存性を向上させた.佐藤らは,勾配ベースのS1スコアとSSIMが人間の主観的な「類似性」の判断をよく再現することを統計的に実証した\cite{SATOTakuto20212021-047}.さらに,Winderlichらは改良SSIMとmedoid表現を組み合わせ,クラス分離性と代表性を両立させる手法を提案している\cite{winderlich2024classification}.

本研究では,Batch-SOMフレームワーク上で8つの距離指標を厳密に比較する.既存手法としてS1,SSIM5,Euclidean,新たに提案する単一手法としてKAPPA,GSSIM,そして複数の指標を統合した複合手法を同一条件で評価する.従来のcentroid表現に代わり,medoidおよびtrue medoidを用いることで解釈性を向上させ,学習期(1991–1997)と独立検証期(1998–2000)を明確に分離して汎化性能を検証する.

\section{従来手法}
\subsection{バッチ版SOM}
自己組織化マップ(SOM)\cite{kohonen1990self}は教師なし学習によって高次元データを低次元(通常2次元)マップに写像する手法である.オンライン版SOMは1サンプルずつ処理するが,バッチ版SOM(Batch-SOM)は全データまたはミニバッチ単位で処理することで計算効率と収束安定性を向上させる.

本研究では,Graphics Processing Unit(GPU)によるミニバッチ版バッチSOMを実装した\cite{vettigliminisom}.学習アルゴリズムは以下の通りである:

\begin{enumerate}
\item \textbf{初期化}:$M = m_x \times m_y$個のノード(ニューロン)を2次元格子上に配置し,各ノードの重みベクトル$\bm{w}_j \in \mathbb{R}^D$($j=1,...,M$)をランダムまたはデータからのサンプリングで初期化する.
\item \textbf{ミニバッチ処理}:各反復$t$において,データをサイズ$B$のミニバッチに分割して処理する.
\item \textbf{BMU探索}:各入力$\bm{x}_i$に対して,最良一致ユニット(Best Matching Unit; BMU)を距離関数$d(\cdot, \cdot)$により決定:
\begin{equation}
c(i) = \arg\min_{j} d(\bm{x}_i, \bm{w}_j)
\end{equation}
\item \textbf{近傍関数}:BMU $c(i)$と各ノード$j$間の近傍関数$h_{c(i),j}$を計算する.本研究ではガウス関数を使用:
\begin{equation}
h_{c(i),j} = \exp\left(-\frac{\|\bm{r}_{c(i)} - \bm{r}_j\|^2}{2\sigma(t)^2}\right)
\end{equation}
ここで$\bm{r}_j$はノード$j$のグリッド座標,$\sigma(t)$は学習進行に応じて減衰する近傍幅である.
\item \textbf{重み更新}:ミニバッチ内の全サンプルについて分子・分母を累積し,一括更新:
\begin{equation}
\bm{w}_j^{new} = \frac{\sum_{i \in \text{batch}} h_{c(i),j} \bm{x}_i}{\sum_{i \in \text{batch}} h_{c(i),j}}
\end{equation}
\item \textbf{近傍幅の減衰}:$\sigma(t)$は学習全体の進行に応じて単調減少させる.本研究では漸近減衰を採用:
\begin{equation}
\sigma(t) = \frac{\sigma_0}{1 + t/(T/2)}
\end{equation}
ここで$\sigma_0$は初期近傍幅,$T$は総反復回数である.
\end{enumerate}

この枠組みでは,距離関数$d(\cdot, \cdot)$の選択がBMU決定,ひいては学習結果全体に決定的な影響を与える.以下では各距離関数を定義する.

\subsection{距離関数}
バッチSOMおよび気象の気圧配置に関しては,既存手法として以下の距離関数が用いられている.

\paragraph{Euclidean(ユークリッド距離)}
最も基本的な距離で,値の差の総量を測る:
\begin{equation}
d_{\mathrm{EUC}}(x,w)=\sqrt{\sum_{s\in\Omega}\bigl(x(s)-w(s)\bigr)^2+\varepsilon}
\end{equation}

\paragraph{S1(Teweles-Wobus Score)}
一次差分(勾配)の相対誤差に基づき,前線帯の鋭さ・広がり・配置の一致を測る.加法定数に不変:
\halfeq{%
d_{\mathrm{S1}}(x,w)=100\times
\frac{\sum\limits_{(i,j)}|\Delta_x^x(i,j)-\Delta_x^w(i,j)|+\sum\limits_{(i,j)}|\Delta_y^x(i,j)-\Delta_y^w(i,j)|}
{\sum\limits_{(i,j)}\max(|\Delta_x^x(i,j)|,|\Delta_x^w(i,j)|)+\sum\limits_{(i,j)}\max(|\Delta_y^x(i,j)|,|\Delta_y^w(i,j)|)+\varepsilon}%
}
ここで$\Delta_x^x(i,j)=x(i,j+1)-x(i,j)$,$\Delta_y^x(i,j)=x(i+1,j)-x(i,j)$,$w$側も同様である.

\paragraph{SSIM5(構造類似度指標)}
$5\times 5$の局所窓で平均・分散・共分散を計算し,明るさ・コントラスト・構造の一致度を測る:
\halfeq{%
d_{\mathrm{SSIM5}}(x,w)=1-\frac{1}{|\Omega|}\sum_{s\in\Omega}
\frac{\bigl(2\,\mu_x(s)\mu_w(s)\bigr)\bigl(2\,\operatorname{cov}_{xw}(s)\bigr)}
{\bigl(\mu_x(s)^2+\mu_w(s)^2\bigr)\bigl(\sigma_x^2(s)+\sigma_w^2(s)\bigr)+\varepsilon}%
}
$\mu,\sigma^2,\operatorname{cov}$は$5\times 5$移動窓内の局所統計量,境界はreflectパディングである.

\section{提案手法}
本研究では,各データ点は入力を各時刻のSLP偏差(hPa)をベクトル化したものとする.$x,w$はそれぞれ入力パターンとノード重みの2次元場($H\times W$),格子領域を$\Omega$とする.$\varepsilon=10^{-12}$は数値安定化のための微小正値である.

このデータ点に対して,新たに以下の距離関数を提案する.

\subsection{距離関数}
\paragraph{GSSIM(勾配構造類似度)}
勾配強度と方向($\cos\theta$)の一致度を,強いエッジほど重く評価:
\halfeq{%
d_{\mathrm{GS}}(x,w)=1-
\frac{\sum\limits_{i,j} \max(G^x,G^w)\;
\frac{2G^xG^w}{(G^x)^2+(G^w)^2+\varepsilon}\;
\tfrac{1}{2}\!\left(1+\frac{g_x^x g_x^w+g_y^x g_y^w}{G^x G^w+\varepsilon}\right)
}{\sum\limits_{i,j}\max(G^x,G^w)+\varepsilon}%
}
$g_x^\cdot,g_y^\cdot$は内部格子の一次差分,$G^\cdot=\sqrt{(g_x^\cdot)^2+(g_y^\cdot)^2+\varepsilon}$である.

本研究で提案するGSSIM(勾配構造類似度)は,画像処理分野における構造類似度指標(SSIM)の概念を勾配場に拡張したものである.第2項の$\frac{2G^xG^w}{(G^x)^2+(G^w)^2}$は勾配強度の類似度を評価し,第3項の$\frac{1}{2}(1+\cos\theta)$(ここで$\cos\theta = \frac{g_x^x g_x^w+g_y^x g_y^w}{G^x G^w}$)は勾配方向の一致度を評価する.

この定式化は,エッジ検出理論\cite{canny1986computational,marr1980theory}における勾配の重要性と,ベクトル場の類似度評価におけるコサイン類似度の有効性に基づいている.分子の$\max(G^x,G^w)$による重み付けにより,強い勾配(前線帯や気圧傾度の大きい領域)での一致度をより重視する設計となっている.

気象場において勾配の強度と方向を同時に評価し,かつ強いエッジを重み付きで評価する距離関数は既存研究には見当たらない.GSSIMは,総観規模気象場の前線構造や気圧傾度の空間パターンを効果的に捉える新規な距離指標である.

\paragraph{KAPPA(曲率距離)}
正規化勾配の発散(曲率)から等圧線の曲がり(凹凸・渦の閉じ具合)を比較:
\halfeq{%
d_{\mathrm{KA}}(x,w)=\tfrac{1}{2}\times
\frac{\sum\limits_{i,j}\bigl|\kappa(x)(i,j)-\kappa(w)(i,j)\bigr|}
{\sum\limits_{i,j}\max\!\bigl(|\kappa(x)(i,j)|,|\kappa(w)(i,j)|\bigr)+\varepsilon}%
}
$\kappa(z)=\nabla\cdot\left(\nabla z/(\|\nabla z\|+\varepsilon)\right)$は中心差分の内部共通格子で評価される.

本研究で用いる曲率$\kappa(z)=\nabla\cdot(\nabla z/(\|\nabla z\|+\varepsilon))$は,水平集法における曲線進展の数理定式化\cite{osher1988fronts}や,微分幾何学における等値線曲率の定義\cite{docarmo1976differential}に基づくものである.

さらに,本研究で提案する距離関数における正規化の形式(差分の絶対値を総和し,最大値の総和で割る構造)は,Dice係数\cite{dice1945measures},Tanimoto係数\cite{tanimoto1957elementary},およびNDVI\cite{rouse1974monitoring}などの既存指標に類似する.

しかしながら,これら既存の指標はいずれも強度値や集合の差異を対象としており,気象場における曲率(正規化勾配の発散)を直接距離関数として定義し,格子全体で比較する手法は報告されていない.したがって,KAPPA(曲率距離)は,既存の正規化手法の数式構造を踏まえつつも,対象量として曲率場を扱う点において新規性を有する.

\subsection{複合手法}
複合手法として,異なる物理的意味を持つ距離指標を統合する新しい手法を提案する.本研究の新規性は,複数の距離関数をRoot Mean Square(RMS)で統合する点にある.

\paragraph{RMS統合の理論的根拠}
単純な算術平均や重み付き和ではなくRMSを採用した理由は以下の通りである:

\begin{enumerate}
\item \textbf{スケール不変性}:各距離指標は異なる物理量(勾配,曲率,構造類似度)を評価するため,そのスケールが大きく異なる.min-max正規化により各指標を$[0,1]$区間に標準化することで,スケールの違いを吸収する.

\item \textbf{ユークリッド的統合}:RMSは多次元空間における距離の自然な統合方法である.各正規化された距離を独立な次元と見なすと,RMSは多次元空間でのユークリッド距離に相当する:
\begin{equation}
d_{\text{RMS}} = \sqrt{\frac{1}{n}\sum_{i=1}^{n} \tilde{d}_i^2}
\end{equation}

\item \textbf{外れ値への感度}:算術平均と比較して,RMSは大きな値により敏感に反応する.これにより,いずれかの指標で大きな違いが検出された場合,統合距離もその違いを適切に反映する.

\item \textbf{係数フリー}:正規化により各指標が同じオーダー($[0,1]$区間)になるため,重み係数を調整することなく等価的に統合できる.これは実用上の大きな利点である.
\end{enumerate}

\paragraph{行方向正規化の意味}
入力$x$を固定し,全候補$w'$に対してmin-max正規化することで,各入力に対する相対的な類似性を評価する.これにより,絶対的な距離値ではなく,候補間の相対的な順位関係に基づくBMU選択が可能となる.

\paragraph{S1,KA}
S1(勾配の鋭さ)とKAPPA(曲率)を統合:

\halfeq{%
\tilde d_{\mathrm{S1}}(x,w)=\frac{d_{\mathrm{S1}}(x,w)-\min_{w'}d_{\mathrm{S1}}(x,w')}{\max_{w'}d_{\mathrm{S1}}(x,w')-\min_{w'}d_{\mathrm{S1}}(x,w')+\varepsilon}}
\halfeq{%
\tilde d_{\mathrm{KA}}(x,w)=\frac{d_{\mathrm{KA}}(x,w)-\min_{w'}d_{\mathrm{KA}}(x,w')}{\max_{w'}d_{\mathrm{KA}}(x,w')-\min_{w'}d_{\mathrm{KA}}(x,w')+\varepsilon}}
\begin{align}
d_{\mathrm{S1,KA}}(x,w)&=\sqrt{\frac{\tilde d_{\mathrm{S1}}(x,w)^2+\tilde d_{\mathrm{KA}}(x,w)^2}{2}}
\end{align}

\paragraph{GS,KA}
GSSIM(エッジの強さ・向き)とKAPPA(曲率)を統合:
\begin{equation}
d_{\mathrm{GS,KA}}(x,w)=\sqrt{\frac{d_{\mathrm{GS}}(x,w)^2+\tilde d_{\mathrm{KA}}(x,w)^2}{2}}
\end{equation}

\paragraph{S1,GS,KA}
S1(鋭さ)・GSSIM(方向)・KAPPA(曲率)の三要素を総合:
\halfeq{%
d_{\mathrm{S1,GS,KA}}(x,w)=\sqrt{\frac{\tilde d_{\mathrm{S1}}(x,w)^2+d_{\mathrm{GS}}(x,w)^2+\tilde d_{\mathrm{KA}}(x,w)^2}{3}}
}

\subsection{Medoid表現}
SOMの各ノードの代表パターンとして,従来はcentroid(割り当てられたサンプルの平均)が使用されるが,平均化により勾配が平滑化される問題がある.そこで本研究では,各ノードに割り当てられたインデックス集合$I_c$に対し,ノード内総距離
\begin{equation}
\mathrm{cost}(i)=\sum_{j\in I_c} d\bigl(X_j, X_i\bigr)
\end{equation}
が最小となる実サンプル$X_{i^\star}$($i^\star=\arg\min_{i\in I_c}\mathrm{cost}(i)$)をそのノードのtrue medoidとする.これにより実データの鋭さを保持した代表パターンが得られる.

\subsection*{数式内の変数説明}
\begin{itemize}
\item $x,w$:入力パターンとノード重み($H\times W$の格子場)
\item $\Omega$:格子領域の添字集合,$|\Omega|$は格子点数
\item $\mu_x(s),\mu_w(s)$:画素$s$近傍($5\times 5$)での局所平均
\item $\sigma_x^2(s),\sigma_w^2(s)$:同局所分散
\item $\operatorname{cov}_{xw}(s)$:同局所共分散
\item $\Delta_x^\cdot,\Delta_y^\cdot$:水平方向・鉛直方向の一次差分
\item $g_x^\cdot,g_y^\cdot$:内部共通格子での勾配成分,$G^\cdot$:勾配強度
\item $\bm{n}$:正規化勾配ベクトル,$\kappa(z)=\nabla\cdot\bm{n}$(中心差分)
\item $\varepsilon$:数値安定化の微小正値($10^{-12}$)
\end{itemize}

\section{実験設定}
\subsection{データと前処理}
\begin{itemize}
\item \textbf{物理量}:海面更正気圧(SLP)
\item \textbf{領域}:日本周辺域(15–55°N, 115–155°E)
\item \textbf{期間}:学習1991–1997年,検証1998–2000年(日次09 Japan Standard Time; JST)
\item \textbf{ラベル}:吉野\cite{吉野2002日本の気候}の「気圧配置ごよみ」準拠
\item \textbf{前処理}:PaからhPaへ変換後,領域平均を引く.
\end{itemize}

\subsection{ラベル体系の詳細}
本研究で用いるラベル体系は,吉野\cite{吉野2002日本の気候}の付録B「気圧配置ごよみ」に準拠する.基本型は以下の6分類(計15サブタイプ)で構成される:

\begin{description}
\item[1. 西高東低冬型] 冬季に特徴的なシベリア高気圧とアリューシャン低気圧に起因する西高東低の場.

\item[2. 気圧の谷型] 低気圧あるいは気圧の谷の通過に対応する型.
  \begin{description}
    \item[A.] 低気圧が北海道またはサハリン付近を東に進む
    \item[B.] 低気圧が日本海から北東に進む
    \item[C.] 低気圧が台湾から日本の太平洋岸を東〜東北東に進む
    \item[D.] 二つ玉低気圧,または日本海と太平洋に低圧部
  \end{description}

\item[3. 移動性高気圧型] 移動性高気圧に覆われる型.
  \begin{description}
    \item[A.] 日本の北方または北部を東に進む
    \item[B.] 日本列島上(主として本州上)を東に進む
    \item[C.] 帯状高気圧
    \item[D.] 日本の太平洋岸または南方を東に進む
  \end{description}

\item[4. 前線型] 停滞性の前線が卓越する型.
  \begin{description}
    \item[A.] 日本列島上をほぼ東西方向に走る主として停滞性の前線
    \item[B.] 太平洋岸または日本南方をほぼ東西方向に走る主として停滞性の前線
  \end{description}

\item[5. 南高北低夏型] 原則として北太平洋高気圧が日本列島を支配する夏型.

\item[6. 台風型] 台風の位置に応じて次のサブタイプに区分する.
  \begin{description}
    \item[A.] 台風が南九州より南方の海上にある場合
    \item[B.] 台風が本州およびその接岸地帯にある場合
    \item[C.] 台風が北日本にある場合
  \end{description}
\end{description}

\begin{figure*}[t!]
\centering
\begin{tabular}{ccc}
\includegraphics[width=0.30\textwidth]{fig/slp_anom_19910109.png} &
\includegraphics[width=0.30\textwidth]{fig/slp_anom_19941001.png} &
\includegraphics[width=0.30\textwidth]{fig/slp_anom_19920406.png} \\
(a) 西高東低冬型(ラベル:1) & (b) 気圧の谷型(ラベル:2A) & (c) 移動性高気圧型(ラベル:3A) \\
\includegraphics[width=0.30\textwidth]{fig/slp_anom_19930707.png} &
\includegraphics[width=0.30\textwidth]{fig/slp_anom_19990728.png} &
\includegraphics[width=0.30\textwidth]{fig/slp_anom_19910819.png} \\
(d) 前線型(ラベル:4A) & (e) 南高北低夏型(ラベル:5) & (f) 台風型(ラベル:6A) \\
\end{tabular}
\caption{基本型の代表的な海面更正気圧偏差パターン(単位:hPa).各パネルは日本周辺域(15–55°N, 115–155°E)における領域平均からの偏差を示す.}
\label{fig:basic_patterns}
\end{figure*}

この他に,移行型(例:「3A - 2D」)と複合型(例:「2A + 2C」)が定義されているが,本研究では基本型15種のみを評価対象とした.詳細な定義は吉野\cite{吉野2002日本の気候}を参照されたい.

\subsection{SOM設定と学習条件}
\begin{itemize}
\item マップサイズ:$10\times 10$
\item 反復回数:$1000$,バッチサイズ:$128$
\item 近傍関数:ガウス関数,初期$\sigma=2.5$(学習全体で減衰)
\item 学習率:$0.5$
\end{itemize}

\subsection{評価指標}
\begin{itemize}
\item Macro Recall(基本ラベル):各基本ラベルの再現率の平均.混同行列は「基本ラベル vs クラスタ列」で集計する.
\item NodewiseMatchRate(基本):各ノードの多数決(基本ラベル)とtrue medoidの基本ラベルの一致率.
\end{itemize}
学習期および独立検証期に対して算出する.

\section{実験結果}
本節では,シード値1〜50で50回試行した集計結果を示す.手法別のMacro Recall(学習・検証)およびNodewiseMatchRate(学習)の要約統計を表\ref{tab:macro_nodewise_summary}に,基本ラベルごとの平均再現率の横断ピボット表を学習(表\ref{tab:pivot_train})と検証(表\ref{tab:pivot_valid})でそれぞれ示す.

50試行において学習期のMacro RecallはS1,KA/S1,GS,KA/GS,KAが0.37 ± 0.02で同等の上位値を示した.検証期ではS1,KAが0.23 ± 0.02でやや高く,実用上はS1,KAが汎化性能において優位と判断できる.NodewiseMatchRate(学習)は複合手法のスコアが低い.

加えて,各手法が「どのくらいのラベルを検出できているか(非ゼロ再現率の基本ラベル数)」に着目すると,学習期ではSSIM5のみが15/15ラベルを網羅し,他の7手法はいずれも14/15(主として6Cが0)にとどまった(表\ref{tab:pivot_train}).一方で検証期は差がより明瞭で,S1が11/15と最も広いカバレッジを示し,S1,GS,KA・GS,KA・GSSIMが10/15,S1,KAが8/15,KAPPA/Euclidean/SSIM5はいずれも7/15にとどまった(表\ref{tab:pivot_valid}).

検出が安定なラベルとしては,1(西高東低冬型),3B(移動性高気圧:本州上),4A(東西前線)が挙げられ,多くの手法で検証期でも高い再現率が維持される.一方で,2C(太平洋岸低気圧),3C(帯状高気圧),3D(南岸高気圧),台風系(6A/6C)は複数手法で検出ゼロになりやすく,事例数の少なさや場の多様性の高さが影響していると解釈できる.

以上より,実運用ではS1,KAを標準距離とする構成が妥当と考える.

\begin{table*}[t]
\centering
\caption{手法別 Macro Recall(学習・検証)と NodewiseMatchRate(学習)の統計量(平均±標準偏差, N=50)}
\label{tab:macro_nodewise_summary}
\begin{tabular}{lccc}
\hline
Method & Macro Recall(学習) & Macro Recall(検証) & NodewiseMatchRate(学習) \\
\hline
S1,KA      & 0.37 ± 0.02 & 0.23 ± 0.02 & 0.25 ± 0.03 \\
S1,GS,KA   & 0.37 ± 0.02 & 0.22 ± 0.02 & 0.24 ± 0.04 \\
GS,KA      & 0.37 ± 0.02 & 0.21 ± 0.01 & 0.25 ± 0.05 \\
S1         & 0.36 ± 0.02 & 0.21 ± 0.02 & 0.26 ± 0.04 \\
GSSIM      & 0.35 ± 0.02 & 0.21 ± 0.02 & 0.27 ± 0.04 \\
SSIM5      & 0.32 ± 0.02 & 0.17 ± 0.01 & 0.29 ± 0.04 \\
Euclidean  & 0.30 ± 0.02 & 0.18 ± 0.01 & 0.31 ± 0.03 \\
KAPPA      & 0.30 ± 0.01 & 0.21 ± 0.02 & 0.29 ± 0.03 \\
\hline
\end{tabular}
\end{table*}

\begin{table*}[t!]
\centering
\caption{横断ピボット表(学習: 基本ラベルごとの平均再現率)}
\label{tab:pivot_train}
\scriptsize
\setlength{\tabcolsep}{3pt}
\begin{tabular}{lcccccccccccccccccc}
\hline
Method & 1 & 2A & 2B & 2C & 2D & 3A & 3B & 3C & 3D & 4A & 4B & 5 & 6A & 6B & 6C & Overall & N\_lab \\
\hline
S1,KA      & 0.92 & 0.26 & 0.24 & 0.14 & 0.60 & 0.21 & 0.79 & 0.05 & 0.01 & 0.68 & 0.35 & 0.59 & 0.61 & 0.10 & 0.00 & 0.37 & 14 \\
S1,GS,KA   & 0.91 & 0.35 & 0.27 & 0.07 & 0.61 & 0.20 & 0.79 & 0.08 & 0.02 & 0.66 & 0.34 & 0.59 & 0.45 & 0.17 & 0.00 & 0.37 & 14 \\
GS,KA      & 0.90 & 0.32 & 0.27 & 0.07 & 0.62 & 0.21 & 0.79 & 0.07 & 0.02 & 0.66 & 0.35 & 0.63 & 0.41 & 0.17 & 0.00 & 0.37 & 14 \\
S1         & 0.91 & 0.29 & 0.21 & 0.13 & 0.61 & 0.22 & 0.77 & 0.05 & 0.02 & 0.66 & 0.37 & 0.54 & 0.49 & 0.20 & 0.00 & 0.36 & 14 \\
GSSIM      & 0.91 & 0.27 & 0.18 & 0.09 & 0.59 & 0.19 & 0.75 & 0.06 & 0.02 & 0.63 & 0.31 & 0.55 & 0.56 & 0.18 & 0.00 & 0.35 & 14 \\
SSIM5      & 0.90 & 0.13 & 0.07 & 0.17 & 0.45 & 0.22 & 0.70 & 0.06 & 0.16 & 0.59 & 0.36 & 0.44 & 0.42 & 0.14 & 0.04 & 0.32 & 15 \\
Euclidean  & 0.91 & 0.22 & 0.12 & 0.05 & 0.48 & 0.35 & 0.74 & 0.05 & 0.14 & 0.57 & 0.22 & 0.33 & 0.34 & 0.06 & 0.00 & 0.30 & 14 \\
KAPPA      & 0.88 & 0.25 & 0.22 & 0.10 & 0.45 & 0.06 & 0.78 & 0.07 & 0.01 & 0.68 & 0.26 & 0.62 & 0.04 & 0.04 & 0.00 & 0.30 & 14 \\
\hline
\end{tabular}
\end{table*}

\begin{table*}[t!]
\centering
\caption{横断ピボット表(検証: 基本ラベルごとの平均再現率)}
\label{tab:pivot_valid}
\scriptsize
\setlength{\tabcolsep}{3pt}
\begin{tabular}{lcccccccccccccccccc}
\hline
Method & 1 & 2A & 2B & 2C & 2D & 3A & 3B & 3C & 3D & 4A & 4B & 5 & 6A & 6B & 6C & Overall & N\_lab \\
\hline
S1,KA      & 0.91 & 0.21 & 0.03 & 0.00 & 0.57 & 0.00 & 0.80 & 0.00 & 0.00 & 0.73 & 0.08 & 0.09 & 0.00 & 0.00 & 0.00 & 0.23 & 8 \\
S1,GS,KA   & 0.92 & 0.20 & 0.02 & 0.00 & 0.45 & 0.02 & 0.80 & 0.00 & 0.00 & 0.64 & 0.08 & 0.12 & 0.00 & 0.01 & 0.00 & 0.22 & 10 \\
GSSIM      & 0.93 & 0.09 & 0.02 & 0.00 & 0.48 & 0.02 & 0.78 & 0.00 & 0.00 & 0.53 & 0.10 & 0.14 & 0.00 & 0.11 & 0.00 & 0.21 & 10 \\
GS,KA      & 0.93 & 0.14 & 0.04 & 0.00 & 0.44 & 0.03 & 0.79 & 0.00 & 0.00 & 0.62 & 0.08 & 0.14 & 0.00 & 0.00 & 0.00 & 0.21 & 10 \\
S1         & 0.95 & 0.04 & 0.01 & 0.00 & 0.53 & 0.01 & 0.77 & 0.00 & 0.00 & 0.65 & 0.06 & 0.08 & 0.02 & 0.04 & 0.00 & 0.21 & 11 \\
KAPPA      & 0.91 & 0.15 & 0.00 & 0.00 & 0.31 & 0.00 & 0.73 & 0.00 & 0.00 & 0.74 & 0.05 & 0.20 & 0.00 & 0.00 & 0.00 & 0.21 & 7 \\
Euclidean  & 0.86 & 0.04 & 0.00 & 0.00 & 0.29 & 0.00 & 0.83 & 0.00 & 0.00 & 0.66 & 0.01 & 0.01 & 0.00 & 0.00 & 0.00 & 0.18 & 7 \\
SSIM5      & 0.86 & 0.00 & 0.00 & 0.01 & 0.28 & 0.00 & 0.78 & 0.00 & 0.00 & 0.52 & 0.08 & 0.05 & 0.00 & 0.00 & 0.00 & 0.17 & 7 \\
\hline
\end{tabular}
\end{table*}

\begin{figure*}[t!]
\centering
\begin{tabular}{cc}
\includegraphics[width=0.45\textwidth]{fig/s1k_som_node_avg_all.png} &
\includegraphics[width=0.45\textwidth]{fig/s1k_som_node_true_medoid_all.png} \\
(a) centroid(平均; 雪だるま現象が顕著) & (b) true medoid(実サンプル代表) \\
\end{tabular}
\caption{S1,KA 手法における最良学習(seed=20)の $10\times 10$ SOM 出力マップ.右図(true medoid)では,ノード多数決ラベルと medoid の基本ラベルが一致したノードに色付けしている(凡例:1=青,2D=黄色,3B=オレンジ,4A=ピンク,4B=紫,5=赤,6B=緑).}
\label{fig:s1k_maps}
\end{figure*}

図\ref{fig:s1k_maps}に,S1,KA 手法における学習精度が最も高かったseed=20のSOM出力($10\times 10$)を示す.左は centroid(平均)であり,平均化の影響により低気圧中心や前線の鋭さが弱まり,いわゆる「雪だるま現象」によりノードの気圧配置に対するラベル付けが困難となる.特に台風(6系)では極小が鈍化し識別がほぼ不可能である.一方,右のtrue medoidでは実データの鋭さ・位相が保持され,ノード多数決ラベルとmedoidの基本ラベルが一致したノードを色付け(1=青,2D=黄色,3B=オレンジ,4A=ピンク,4B=紫,5=赤,6B=緑)することで,専門家の手を借りずとも各ノードに対して一貫したラベル付けが可能となることが期待される.

\section{考察}
\subsection{手法の特性と補完性}
S1は勾配差の比率に基づき前線帯や冬型で有効である.特に等圧線の密集度や前線の鋭さを定量化する点で気象学的に妥当性が高い.SSIM5は局所構造(形状・位置・コントラスト)を評価し,人間の視覚認知と整合する.GSSIMは勾配強度と方向の両面を重み付きで評価し,エッジ検出の観点から構造を捉える.KAPPAは等圧面の曲率構造に注目し,高低気圧の閉じた構造や渦度場の特徴を抽出する.

複合手法(S1,KA/GS,KA/S1,GS,KA)はこれらの利点をRMS統合で併合し,BMU探索における汎化を支える.特にS1,KAは勾配(前線の鋭さ)と曲率(高低気圧の構造)の両面を捉えることで,総観規模現象の本質的特徴を包括的に評価できる.

\subsection{Medoid表現の有効性}
centroidは平均化により勾配が平滑化されやすい一方,medoidは実データ実現値を代表として保持し,解釈性と検索性を高める.本実装では標準でtrue medoidを算出する.

\subsection{汎化性能の解釈}
S1は勾配差に敏感,KAPPAは曲率(幾何)を評価するため,S1,KAは形状と鋭さを同時に捉えBMU探索の頑健性を高める.その結果,検証Macro Recallで最良を示した.一方,NodewiseMatchRateではEuclideanが最大であり,ノード内多数決とmedoidラベルの一致を生みやすい(振幅差に整合).用途により,識別(Macro Recall)優先か,ノード代表性(一致率)優先かで距離の使い分けが必要である.

\subsection{ラベル不均衡と評価設計}
台風系(6A/6B/6C)は出現頻度が極端に低く,さらに複合場が約半数を占めるため,検証で再現率が低下しやすい.季節別SOM(冬・梅雨・台風季)や階層的分類(大局型→サブタイプ),外部事象フラグの併用(台風客観解析)により稀少クラスの頑健性を補うことができる.

\section{まとめ}
Batch-SOM上で,既存手法(S1,SSIM5,Euclidean)と提案手法(KAPPA, GSSIM, S1, KA, GS, KA, S1, GS, KA)の8手法を同一条件で比較できる実験枠組みを構築した.実装に準拠した厳密な数式定義を提示し,評価は基本ラベルに対するMacro Recallとノード代表一致率で実施した.centroidの代わりにmedoidを採用することで代表パターンの鋭さを保持できる.S1,KAのスコアが高く実務ではS1,KAの運用が望ましい.今後は多変量拡張,季節別最適化,および循環統計の総合評価(例:Jensen–Shannon距離)への発展を予定する.

本研究は,気圧配置分類の実用化に向けた重要な一歩であり,気象予報支援システムや気候変動影響評価への応用が期待される.

\bibliographystyle{sieicej}
\bibliography{ref}
\end{document}
